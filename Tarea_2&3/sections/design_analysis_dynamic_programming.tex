\epigraph{\textit{Dynamic programming is not about filling in tables. It's about smart recursion!}}{\citeauthor{algorithms_erickson}, \citeyear{algorithms_erickson} \cite{algorithms_erickson}}


\subsubsection{Descripción de la solución recursiva}

La solución por programación dinámica se implementa utilizando una estrategia de tabulación (\textit{Bottom\-Up}), que calcula de forma iterativa los costos mínimos necesarios para transformar una cadena \( S1 \) en otra \( S2 \). Este enfoque optimiza el cálculo al evitar la recalculación de subproblemas, almacenando los resultados parciales en una matriz de costos.

La matriz de costos tiene dimensiones \( N \times M \), donde \( N \) es la longitud de \( S1 \) y \( M \) es la longitud de \( S2 \). Cada celda \( dp[i][j] \) en la matriz representa el costo mínimo para transformar el prefijo o substring \( S1[0..i-1] \) en el prefijo \( S2[0..j-1] \). Este costo se calcula considerando cuatro operaciones posibles: \textit{Inserción}, \textit{Eliminación}, \textit{Sustitución}, 
\textit{Transposición} (si es que es aplicable).

\paragraph{Inicialización}
Se definen los valores iniciales para los casos base:
\begin{itemize}
    \item \( dp[i][0] \): El costo de transformar los primeros \( i \) caracteres de \( S1 \) en una cadena vacía es simplemente el costo acumulado de eliminar esos \( i \) caracteres.
    \item \( dp[0][j] \): El costo de transformar una cadena vacía en los primeros \( j \) caracteres de \( S2 \) es el costo acumulado de insertar esos \( j \) caracteres.
\end{itemize}

\paragraph{Relleno de la Matriz}
La matriz se rellena iterativamente siguiendo el orden \( dp[i][j] \), donde cada celda considera las cuatro operaciones posibles:
\begin{enumerate}
    \item \textbf{Inserción:} El costo de insertar \( S2[j-1] \) en \( S1[0..i-1] \) y sumar el costo mínimo previo \( dp[i][j-1] \).
    \item \textbf{Eliminación:} El costo de eliminar \( S1[i-1] \) y sumar el costo mínimo previo \( dp[i-1][j] \).
    \item \textbf{Sustitución:} Si \( S1[i-1] \neq S2[j-1] \), el costo de sustituir \( S1[i-1] \) por \( S2[j-1] \), más el costo previo \( dp[i-1][j-1] \). Si los caracteres son iguales, no hay costo adicional.
    \item \textbf{Transposición:} Si los caracteres adyacentes entre \( S1 \) y \( S2 \) están intercambiados (\( S1[i-1] = S2[j-2] \) y \( S1[i-2] = S2[j-1] \)), se considera el costo de transponerlos, sumado al costo previo \( dp[i-2][j-2] \).
\end{enumerate}

\paragraph{Propagación de Costos}
El cálculo en cada celda asegura que el valor almacenado en \( dp[i][j] \) sea el mínimo entre los costos de las operaciones disponibles. Este proceso garantiza que:
\begin{itemize}
    \item Cada subproblema (transformar un prefijo de \( S1 \) en un prefijo de \( S2 \)) se resuelve óptimamente antes de ser usado en cálculos posteriores.
    \item Al finalizar, el valor \( dp[N][M] \) en la esquina inferior derecha de la matriz contiene el costo mínimo total para transformar \( S1 \) en \( S2 \).
\end{itemize}

\paragraph{Ejemplo}
Supongamos \( S1 = \texttt{'amores'} \) y \( S2 = \texttt{'amaras'} \):
\begin{itemize}
    \item Se inicializan los casos base: costos acumulados para transformar una cadena en vacío.
    \item En cada celda, se consideran las operaciones mencionadas y se selecciona el mínimo costo.
    \item La matriz resultante tiene el costo total mínimo \( dp[6][6] \), que es la distancia de edición extendida entre las dos cadenas.
\end{itemize}

Este enfoque aprovecha la naturaleza recursiva del problema mientras reduce la redundancia mediante almacenamiento en la matriz, logrando una complejidad temporal de \( O(N \cdot M) \) y una complejidad espacial de \( O(N \cdot M) \).


\subsubsection{Relación de recurrencia}

La relación de recurrencia establece el costo mínimo para transformar un prefijo de \texttt{S1} en un prefijo de \texttt{S2}. Se define como:
\[
D(i, j) = \min \begin{cases} 
      D(i-1, j) + \text{COSTO\_DEL}(\texttt{S1}[i-1]) & \text{(eliminación)} \\
      D(i, j-1) + \text{COSTO\_INS}(\texttt{S2}[j-1]) & \text{(inserción)} \\
      D(i-1, j-1) + \text{COSTO\_SUB}(\texttt{S1}[i-1], \texttt{S2}[j-1]) & \text{(sustitución)} \\
      D(i-2, j-2) + \text{COSTO\_TRANS}(\texttt{S1}[i-2], \texttt{S1}[i-1]) & \text{(transposición)}
   \end{cases}
\]
con los casos base:
\[
D(i, 0) = i \cdot \text{COSTO\_DEL}(\texttt{S1}[i-1])
\]
\[
D(0, j) = j \cdot \text{COSTO\_INS}(\texttt{S2}[j-1])
\]
\\
\subsubsection{Identificación de subproblemas}

Cada subproblema consiste en calcular el costo mínimo de transformar un prefijo de \texttt{S1} en un prefijo de \texttt{S2}. Esto se traduce en encontrar la mínima cantidad de operaciones para cada prefijo de longitud \( i \) y \( j \) de \texttt{S1} y \texttt{S2}, respectivamente. Al resolver estos subproblemas y almacenarlos en la matriz, se evita la recalculación redundante y se optimiza el cálculo de la solución final.

\subsubsection{Estructura de datos y orden de cálculo}

La estructura de datos utilizada es una matriz bidimensional \( D[i][j] \), donde \( D[i][j] \) almacena el costo mínimo para transformar el prefijo de longitud \( i \) de \texttt{S1} en el prefijo de longitud \( j \) de \texttt{S2}. La matriz se rellena secuencialmente desde la esquina superior izquierda hasta la esquina inferior derecha, asegurando que cada valor \( D[i][j] \) dependa únicamente de valores previamente calculados.

\subsubsection{Algoritmo utilizando programación dinámica}

\begin{algorithm}[H]
    \SetKwProg{myproc}{Procedure}{}{}
    \SetKwFunction{DistanciaMinima}{DistanciaMinima}
    
    \DontPrintSemicolon
    \footnotesize

    % Definición del algoritmo principal
    \myproc{\DistanciaMinima{S1, S2}}{
        $n \leftarrow \text{longitud de } S1$\;
        $m \leftarrow \text{longitud de } S2$\;
        Crear matriz $D$ de tamaño $(n+1) \times (m+1)$\;
        
        \For{$i \leftarrow 0$ \KwTo $n$}{
            $D[i][0] \leftarrow i \times \text{COSTO\_DEL}(S1[i-1])$\;
        }
        \For{$j \leftarrow 0$ \KwTo $m$}{
            $D[0][j] \leftarrow j \times \text{COSTO\_INS}(S2[j-1])$\;
        }
        
        \For{$i \leftarrow 1$ \KwTo $n$}{
            \For{$j \leftarrow 1$ \KwTo $m$}{
                $costo\_del \leftarrow D[i-1][j] + \text{COSTO\_DEL}(S1[i-1])$\;
                $costo\_ins \leftarrow D[i][j-1] + \text{COSTO\_INS}(S2[j-1])$\;
                $costo\_sub \leftarrow D[i-1][j-1] + \text{COSTO\_SUB}(S1[i-1], S2[j-1])$\;
                \uIf{$i > 1 \land j > 1 \land S1[i-1] = S2[j-2] \land S1[i-2] = S2[j-1]$}{
                    $costo\_trans \leftarrow D[i-2][j-2] + \text{COSTO\_TRANS}(S1[i-2], S1[i-1])$\;}
                \Else{
                    $costo\_trans \leftarrow \infty$\;
                }
                
                $D[i][j] \leftarrow \min(costo\_del, costo\_ins, costo\_sub, costo\_trans)$\;
            }
        }
        \Return $D[n][m]$\;
    }
    \caption{Algoritmo de Programación Dinámica para la distancia mínima de edición entre dos cadenas.}
    \label{alg:mi_algoritmo_pd}
\end{algorithm}
