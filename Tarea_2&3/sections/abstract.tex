En este informe se aborda el problema de la distancia mínima de edición extendida entre cadenas de caracteres, implementando y evaluando dos enfoques algorítmicos: \textit{Fuerza Bruta} y \textit{Programación Dinámica}. Se presentan las características principales de cada enfoque, junto con una comparación detallada de sus rendimientos en términos de tiempo de ejecución y uso de memoria, utilizando datasets diseñados específicamente para evaluar casos de transposición, semiordenados y desordenados.

La infraestructura experimental incluye la generación automatizada de datos mediante Python y la evaluación de los algoritmos implementados en C++ bajo un entorno controlado. Los resultados obtenidos destacan la ineficiencia del enfoque de \textit{Fuerza Bruta} para cadenas de longitud moderada o alta, en contraste el enfoque de \textit{Programación Dinámica}.
Este estudio subraya la importancia de seleccionar algoritmos adecuados para problemas computacionales complejos, proporcionando una base sólida para abordar problemas similares en áreas como la bioinformática y la corrección de texto.