Los resultados obtenidos evidencian de manera clara las diferencias de rendimiento entre los enfoques de \textit{Fuerza Bruta} y \textit{Programación Dinámica} al resolver el problema de la distancia mínima de edición. El enfoque de \textit{Fuerza Bruta} muestra un crecimiento exponencial en el tiempo de ejecución a medida que aumenta la longitud de S1, alcanzando tiempos inviables para tamaños moderadamente grandes. Por otro lado, \textit{Programación Dinámica} mantiene un rendimiento altamente eficiente, con tiempos de ejecución que permanecen casi constantes, incluso para cadenas largas.

Estos resultados confirman que \textit{Programación Dinámica} no solo es significativamente más rápida, sino que además proporciona una solución escalable para problemas similares en contextos prácticos. Esto refuerza la importancia de elegir algoritmos óptimos al enfrentar problemas que implican un alto costo computacional, ya que un enfoque eficiente puede marcar la diferencia entre la viabilidad o inviabilidad de una solución.